\begin{questions}
\printanswers

\question What is the value of \cc{result} when the following code is executed?

\begin{lstlisting}[basicstyle=\large]
	int result = 29 % 5;
\end{lstlisting}

\begin{solution}
4
\end{solution}

\question What is the value of \cc{result} when the following code is executed?

\begin{lstlisting}[basicstyle=\large]
	int result = 17 / 5;
\end{lstlisting}

\begin{solution}
3
\end{solution}

\question What is the value of \cc{result} when the following code is executed?

\begin{lstlisting}[basicstyle=\large]
	float result = 17 / 5;
\end{lstlisting}

\begin{solution}
3 (or 3.0)
\end{solution}

\question What is the value of \cc{result} when the following code is executed?

\begin{lstlisting}[basicstyle=\large]
	float result = 17 / 5.0;
\end{lstlisting}

\begin{solution}
3.4
\end{solution}

\question What is the value of \cc{result} when the following code is executed?

\begin{lstlisting}[basicstyle=\large]
	boolean result = 5 > 3 && 6 < 4 && 3 == 3;
\end{lstlisting}

\begin{solution}
false
\end{solution}

\question What is the value of \cc{result} when the following code is executed?

\begin{lstlisting}[basicstyle=\large]
	boolean result = (true && true) || (true && false);
\end{lstlisting}

\begin{solution}
true
\end{solution}

\question What is the value of \cc{result} when the following code is executed?

\begin{lstlisting}[basicstyle=\large]
	boolean e1 = (15/4 == 3);
	boolean e2 = (6%4 == 2);
	boolean result = e1 && e2;
\end{lstlisting}

\begin{solution}
true (e1 and e2 are both true)
\end{solution}

\question What is the value of \cc{result} when the following code is executed?

\begin{lstlisting}[basicstyle=\large]
	boolean result = 5 > 3 || 6 < 4 && 3 == 3;
\end{lstlisting}

\begin{solution}
true
\end{solution}

\question Write an expression that adds 5 to the product of 2 and 7.

\begin{solution}
\begin{lstlisting}
2 * 7 + 5 (or 5 + 2 * 7)
\end{lstlisting}
\end{solution}

\question Write an expression that multiplies the sum of 2 and 7 by 5.

\begin{solution}
\begin{lstlisting}	
5 * (2 + 7)
\end{lstlisting}
\end{solution}

\question Write an expression that evaluates to the last digit of a given integer \texttt{n} (assume the given integer is not negative).

\begin{solution}
\begin{lstlisting}	
n % 10
\end{lstlisting}
\end{solution}

\question Write an expression that evaluates to \texttt{true} if a given integer \texttt{n} is between 1 and 6 (including 1 and 6), otherwise evaluates to \texttt{false}.

\begin{solution}
\begin{verbatim}
n >= 1 && n <= 6 
\end{verbatim}
\end{solution}

\question Write an expression that evaluates to \texttt{true} if a given integer is outside the range [1...6] (including 1 and 6), otherwise evaluates to \texttt{false}.

\begin{solution}
\begin{verbatim}
n < 1 || n > 6
or
!(n >= 1 && n <= 6)
\end{verbatim}
\end{solution}


\question Write an assignment statement that assigns, to a \texttt{boolean} variable \texttt{result}, \texttt{true} if two \texttt{boolean} values $e1$ and $e2$ are different, otherwise\texttt{false}. This is known as the \texttt{XOR} operator.

\begin{solution}
\begin{lstlisting}
boolean result = (e1 != e2); //brackets for clarity	
\end{lstlisting}
\end{solution}

\question Write an expression that evaluates to \texttt{true} if a given integer \texttt{n} is a multiple of 3 but not a multiple of 27, otherwise evaluates to \texttt{false}.

\begin{solution}
\begin{lstlisting}	
n % 3 == 0 && n % 27 != 0
\end{lstlisting}
a second way is,
\begin{lstlisting}	
n % 27 == 3 || n% 27 == 9
\end{lstlisting}
\end{solution}

\question Write an assignment statement that assigns, to an integer variable \texttt{result}, the remainder when 57 is divided by 6.

\begin{solution}
\begin{lstlisting}	
int result = 57 % 6;
\end{lstlisting}
\end{solution}

\question \textbf{(Tricky)} Is the following code going to executed successfully? And if so, what is the value of $b$ when the it is executed?

\begin{lstlisting}
float a = 1.5;
int b = 2;
b = b + a;
\end{lstlisting}

\begin{solution}
It does not execute because, in the expression on line 3, right hand side evaluates to a float which cannot be copied into an int variable on the left hand side ($b$).
\end{solution}


\question \textbf{(Tricky)} Is the following code going to executed successfully? And if so, what is the value of $b$ when the it is executed?

\begin{lstlisting}
float a = 1.5;
int b = 2;
b+=a;
\end{lstlisting}

\begin{solution}
It executes successfully because \texttt{b+=a} is equivalent to \texttt{b=(int)(b+a)} and NOT \texttt{b=b+a}. Thus, \texttt{b = (int)(2+1.5) = (int)3.5 = 3}
\end{solution}

\end{questions}