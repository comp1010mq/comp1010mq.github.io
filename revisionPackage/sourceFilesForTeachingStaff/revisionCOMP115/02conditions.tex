\begin{questions}
\question
What is the value of \cc{result} when the following code is executed?

\begin{lstlisting}[basicstyle=\large]
	float result = 1.5;
	int a = 12, b = 5;
	if(b > a) {
		result+=0.3;
	}
	else {
		result-=0.3;
	}
\end{lstlisting}

\begin{solution}
1.2	
\end{solution}  \question
What is the value of \cc{result} when the following code is executed?

\begin{lstlisting}[basicstyle=\large]
	float result = 1.5;
	int a = 12, b = 5;
	if(a % b == a / b) {
		result+=0.3;
	}
	else {
		result-=0.3;
	}
\end{lstlisting}

\begin{solution}
1.8
\end{solution}  \question
What is the value of \cc{result} when the following code is executed, if,

\begin{enumerate}
\item \cc{a = 7, b = 12}
\item \cc{a = 15, b = 12}
\item \cc{a = 12, b = 12}
\end{enumerate}


\begin{lstlisting}[basicstyle=\large]
	int result = 4;
	if(b > a) {
		result = 1;
	}
	else if(b < a) {
		result = -1;
	}
	else {
		result = 0;
	}	
\end{lstlisting}

\begin{solution}
	\item \cc{a = 7, b = 12 -> result = 1}
\item \cc{a = 15, b = 12 -> result = -1}
\item \cc{a = 12, b = 12 -> result = 0}

\end{solution}  \question
For what range of \cc{marks}, will the value of \cc{result} when the following code is executed, be 2?

\begin{lstlisting}[basicstyle=\large]
	int result = 0;
	int marks = (int)random(101); //between 0 and 100
	if(marks < 50)
		result = 0;
	else if(marks < 65)
		result = 1;
	else if(marks < 75)
		result = 2;
	else if(marks < 85)
		result = 3;
	else
		result = 4;
\end{lstlisting}

\begin{solution}
65 to 74
\end{solution}  \question
Assuming the existence of an integer variable \cc{data} with some value stored in it, write a piece of code that assigns the absolute value of \cc{data} into another integer variable \cc{result} 

\begin{solution}
\begin{lstlisting}
	if(data >= 0)
		result = data;
	else
		result = -data;
\end{lstlisting}
\end{solution}  \question
Assuming the existence of two integer variables \cc{a, b} with some values stored in them, write a piece of code that assigns, to a third integer variable \cc{result},

	\begin{enumerate}
	\item 1 if both 	\cc{a, b} are positive
	\item -1 if both \cc{a, b} are negative
	\item 0 in all other cases
	\end{enumerate}

\begin{solution}
\begin{lstlisting}
	int result = 0;
	if(a > 0 && b >  0)
		result = 1;
 	if(a < 0 && b < 0)
 		result = -1;
 	//in all other cases, result
 	//remains unchanged (0)
\end{lstlisting}
\end{solution}  \question
Assuming the existence of two integer variables \cc{a, b} with some values stored in them, write a piece of code that assigns, to a third integer variable \cc{result},

	\begin{enumerate}
	\item 1 if both 	\cc{a, b} are even
	\item -1 if both \cc{a, b} are odd
	\item 0 in all other cases
	\end{enumerate}
	
\begin{solution}
\begin{lstlisting}
	int result = 0;
	if(a % 2 == 0 && b % 2 ==  0)
		result = 1;
 	if(a % 2 != 0 && b % 2 != 0)
 		result = -1;
 	//in all other cases, result
 	//remains unchanged (0)
\end{lstlisting}
\end{solution}  \question
Assuming the existence of an floating-point variable \cc{data} with some value stored in it, write a piece of code that assigns, to a second integer variable \cc{result}, the value of \cc{data} rounded-off to the nearest integer. For example, if \cc{data = 4.6}, \cc{result} should be \cc{5}. If \cc{data = 4.4}, \cc{result} should be \cc{4}.  if \cc{data = 4.5}, \cc{result} should be \cc{5}.  if \cc{data = 4.0}, \cc{result} should be \cc{4}.

\begin{solution}
\begin{lstlisting}
	if(data - (int)data < 0.5)
		result = (int)data; //round down
	else
		result = (int)data + 1; //round up
\end{lstlisting}

another (more compact, but also more cryptic) way,

\begin{lstlisting}
	result = (int)(data + 0.5);
\end{lstlisting}
\end{solution}  \question
Assuming the existence of three integer variables \cc{a, b, c} with some values stored in them, write a piece of code that assigns, to a fourth integer variable \cc{result} according to the following table,

\begin{tabular}{cccc}
  a & b & c & result\\
  \hline
  positive & positive & positive & 0\\
  positive & positive & non-positive & 1\\
  positive & non-positive & positive & 2\\
  positive & non-positive & non-positive & 3\\
  non-positive & positive & positive & 4\\
  non-positive & positive & non-positive & 5\\
  non-positive & non-positive & positive & 6\\
  non-positive & non-positive & non-positive & 7
  \end{tabular}

\begin{solution}
\begin{lstlisting}
if(a > 0)
	if(b > 0)
		if(c > 0)
			result = 0;
		else //c <= 0
			result = 1;
	else //b <= 0
		if(c > 0)
			result = 2;
		else //c <= 0
			result = 3;
else //a <= 0
	if(b > 0)
		if(c > 0)
			result = 4;
		else //c <= 0
			result = 5;
	else //b <= 0
		if(c > 0)
			result = 6;
		else //c <= 0
			result = 7;
\end{lstlisting}
a second way, alas, with more expression checks -

\begin{lstlisting}
if(a > 0 && b > 0 && c > 0)
	result = 0;
if(a > 0 && b > 0 && c <= 0)
	result = 1;
if(a > 0 && b <= 0 && c > 0)
	result = 2;
if(a > 0 && b <= 0 && c <= 0)
	result = 3;
if(a <= 0 && b > 0 && c > 0)
	result = 4;
if(a <= 0 && b > 0 && c <= 0)
	result = 5;
if(a <= 0 && b <= 0 && c > 0)
	result = 6;
if(a <= 0 && b <= 0 && c <= 0)
	result = 7;
\end{lstlisting}
\end{solution}

\end{questions}