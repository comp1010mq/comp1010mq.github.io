\begin{questions}

\question What is the value of \cc{result} when the following code is executed?

\begin{lstlisting}[basicstyle=\large]
	int[] taxicab = {1, 7, 2, 9};
	int result = taxicab.length;
\end{lstlisting}

\begin{solution}
4
\end{solution}

\question What is the value of \cc{result} when the following code is executed?

\begin{lstlisting}[basicstyle=\large]
	int[] taxicab = {1, 7, 2, 9};
	int result = taxicab[1] + taxicab[3];
\end{lstlisting}

\begin{solution}
7 + 9 = 16	
\end{solution}

\question What is the value of \cc{result} when the following code is executed?

\begin{lstlisting}[basicstyle=\large]
	int[] taxicab = {1, 7, 2, 9};
	int result = taxicab[taxicab.length - 1];
\end{lstlisting}

\begin{solution}
9
\end{solution}

\question What is the value of \cc{result} when the following code is executed?

\begin{lstlisting}[basicstyle=\large]
	int[] taxicab = {1, 7, 2, 9};
	int result = 0;
	for(int i = 0; i < taxicab.length; i++) {
		result+=taxicab[i];
	}
\end{lstlisting}

\begin{solution}
1 + 7 + 2 + 9 = 19
\end{solution}

\question What is the state of array \cc{taxicab} when the following code is executed?

\begin{lstlisting}[basicstyle=\large]
	int[] taxicab = {1, 7, 2, 9};
	int result = 0;
	for(int i = 0; i < taxicab.length; i++) {
		taxicab[i]*=2;
	}
\end{lstlisting}

\begin{solution}
$\{2, 14, 4, 18\}$	
\end{solution}

\question What is the state of array \cc{taxicab} when the following code is executed?

\begin{lstlisting}[basicstyle=\large]
int[] taxicab = {1, 7, 2, 9};
int result = 0;
for(int i = 0; i < taxicab.length; i++) {
	if(i % 2 == 0) {
		taxicab[i]*=2;
	}
}
\end{lstlisting}
$\{2, 7, 4, 9\}$	
\begin{solution}
	
\end{solution}

\question What is the state of array \cc{taxicab} when the following code is executed?

\begin{lstlisting}[basicstyle=\large]
int[] taxicab = {1, 8, 6, 10, 9, 5, 7};
int result = 0;
for(int i = 0; i < taxicab.length; i++) {
	if(taxicab[i] % 2 == 0) {
		taxicab[i]/=2;
	}
}
\end{lstlisting}

\begin{solution}
$\{1, 4, 3, 5, 9, 5, 7\}$	
\end{solution}

\question Write a piece of code that declares and instantiates a array that can hold 8000 floating-point values.

\begin{solution}
\begin{lstlisting}
	double[] arr = new double[8000];
\end{lstlisting}
\end{solution}

\question Write a piece of code that declares and instantiates an array \texttt{arr} that can hold 666 boolean values.

\begin{solution}
\begin{lstlisting}
boolean[] arr = new boolean[666];
\end{lstlisting}
\end{solution}

\question Assuming that the array \texttt{arr} holds an array that holds 2000 integers (that is, it has already been declared and instantiated), write a piece of code, that, using a loop, assigns,

	\begin{itemize}
	\item 1 to the first item of the array
	\item 2 to the second item of the array
	\item 3 to the third item of the array
	\item $\ldots$
	\end{itemize}
	
\begin{solution}
\begin{lstlisting}
int[] arr = new int[2000];
for(int i=0; i < arr.length; i++) {
	arr[i] = (i+1);
\end{lstlisting}
\end{solution}

\question Assuming that the array \texttt{arr} holds an array that holds 2000 integers (that is, it has already been declared and instantiated), write a piece of code, that, using a loop, assigns,

	\begin{itemize}
	\item 1 to the first item of the array
	\item 5 to the second item of the array
	\item 9 to the third item of the array
	\item 13 to the fourth item of the array
	\item $\ldots$
	\end{itemize}

\begin{solution}
\begin{lstlisting}
int[] arr = new int[2000];
for(int i=0; i < arr.length; i++) {
	arr[i] = 1 + 4*i;
\end{lstlisting}

another way,

\begin{lstlisting}
int[] arr = new int[2000];
int val = 1;
for(int i=0; i < arr.length; i++) {
	arr[i] = val;
	val+=4;
}
\end{lstlisting}
\end{solution}

\question Assuming that the array \texttt{arr} holds an array that holds $n > 0$ integers (that is, it has already been declared and instantiated), write a piece of code, that, using a loop, assigns,

	\begin{itemize}
	\item $n$ to the first item of the array
	\item $n - 1$ to the second item of the array
	\item $n - 2$ to the third item of the array
	\item $\ldots$
	\item 1 to the last item of the array
	\end{itemize}
	
	Note that you can access the number of items in array \cc{arr} by \cc{arr.length}.
	
\begin{solution}
\begin{lstlisting}
int[] arr = new int[2000];
for(int i=0; i < arr.length; i++) {
	arr[i] = arr.length - i;
}
\end{lstlisting}

another way,

\begin{lstlisting}
int[] arr = new int[2000];
int val = arr.length;
for(int i=0; i < arr.length; i++) {
	arr[i] = val;
	val--;
}
\end{lstlisting}
\end{solution}

\question Consider the following array \cc{arr},

\begin{lstlisting}
	float[] arr=  {-1.2, 2.5, 1.3, 0, 0, 1.7, -1.9, 1.1, 0, 0, 0.6};
\end{lstlisting}

\begin{parts}
\part Write a piece of code that stores in a variable \cc{result}, the number of items in array \cc{arr} that are greater than 1.4.

\begin{solution}
\begin{lstlisting}
int result = 0;
for(int i=0; i < arr.length; i++) {
	if(arr[i] > 1.4) {
		result++;
	}
}
\end{lstlisting}
\end{solution}

\part Write a piece of code that stores in a variable \cc{result}, the number of negative items in array \cc{arr}.

\begin{solution}
\begin{lstlisting}
int result = 0;
for(int i=0; i < arr.length; i++) {
	if(arr[i] < 0) {
		result++;
	}
}
\end{lstlisting}
\end{solution}

\part Write a piece of code that stores in a variable \cc{max}, the highest value stored in array \cc{arr}.
\end{parts}

\begin{solution}
\begin{lstlisting}[style=buggy]
@int max = arr[0];@
for(@int i=1@; i < arr.length; i++) {
	if(arr[i] > max) {
		max = arr[i];
	}
}
\end{lstlisting}

Note that the above method works only when the array has at least one item in it, otherwise generates \texttt{ArrayIndexOutOfBoundsException} if the array was instantiated to an array of size 0, or generates \texttt{NullPointerException} if the array has been initialised to \texttt{null}. The solution below works for \textbf{any} array.

\begin{lstlisting}[style=correct]
int max = Integer.MIN_VALUE; //smallest value possible
if(arr != null) {
	for(@int i=0@; i < arr.length; i++) {
		if(arr[i] > max) {
			max = arr[i];
		}
	}
}
\end{lstlisting}
\end{solution}

\question Assuming that array \cc{arr} hold 20 random integers, write a piece of code that stores in a variable \cc{result},

\begin{itemize}
\item \cc{true} if the array \cc{arr} is sorted in ascending order (such that each item is more than or equal to the previous item).
\item \cc{false} otherwise.	
\end{itemize}

\begin{solution}
\begin{lstlisting}[style=correct]
boolean result = true; //assume to be in ascending order
for(@int i=0; @i < arr.length - 1@; i++) {
	if(arr[i] < arr[i - 1]@) { //violation to ascending order
		result = false;
	}
}
\end{lstlisting}

The above solution will go through the entire array even if the first two items are \texttt{arr[0] = 5, arr[1] = 2}. The following modification makes the loop immediately terminate as soon as the first violation is encountered.

\begin{lstlisting}[style=correct]
boolean result = true; //assume to be in ascending order
for(@int i=0; @result == true && i < arr.length - 1@; i++) {
	if(arr[i] < arr[i - 1]@) { //violation to ascending order
		result = false;
	}
}
\end{lstlisting}
\end{solution}

\question \textbf{(challenging)} Assuming that array \cc{arr} hold 20 random integers, write a piece of code that stores in a variable \cc{allUnique},

\begin{itemize}
\item \cc{true} if every item in the array \cc{arr} is unique.
\item \cc{false} otherwise.	
\end{itemize}

\begin{solution}
\begin{lstlisting}
boolean allUnique = true;
for(int i=0; allUnique == true && i < arr.length; i++) {
	//check if arr[i] exists again
   for(int k=i+1; allUnique == true && k < arr.length; k++) {
     if(arr[i] == arr[k]) {
       allUnique = false;
     }
   }
}
\end{lstlisting}
\end{solution}
\end{questions}