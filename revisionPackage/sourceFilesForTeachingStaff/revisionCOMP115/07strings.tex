\begin{questions}
\question What is the value of \cc{result} when the following code is executed?

\begin{lstlisting}[basicstyle=\large]
	String str = "Aloha";
	int result = str.length()
\end{lstlisting}

\begin{solution}
5
\end{solution}

\question What is the value of \cc{result} when the following code is executed?

\begin{lstlisting}[basicstyle=\large]
	String str = "Hola, amigo!";
	char result = str.charAt(4);
\end{lstlisting}

\begin{solution}
, (comma)
\end{solution}

\question What is the value of \cc{result} when the following code is executed?

\begin{lstlisting}[basicstyle=\large]
	String str = "Hola, amigo!";
	int result = str.indexOf('a');
\end{lstlisting}

\begin{solution}
3	
\end{solution}

\question What is the value of \cc{result} when the following code is executed?

\begin{lstlisting}[basicstyle=\large]
	String str = "Hola, amigo!";
	int result = str.indexOf('A');
\end{lstlisting}

\begin{solution}
-1
\end{solution}

\question What is the value of \cc{result} when the following code is executed?

\begin{lstlisting}[basicstyle=\large]
	String str = "Hola, amigo!";
	String result = str.substring(7, 10);
\end{lstlisting}

\begin{solution}
``mig"
\end{solution}

\question What is the value of \cc{result} when the following code is executed?

\begin{lstlisting}[basicstyle=\large]
	String str = "Yes minister";
	String result = 'S' + str.substring(5);
\end{lstlisting}

\begin{solution}
``Sinister"
\end{solution}

\question What is the value of \cc{result} when the following code is executed?

\begin{lstlisting}[basicstyle=\large]
String str = "She sells sea shells on the sea shore";
int result = 0;
for(int i = 0; i < str.length(); i++) {
	char ch = str.charAt(i);
	if(ch == 's') {
		result++;
	}
}
\end{lstlisting}

\begin{solution}
7
\end{solution}

\question Assuming String \cc{str} contains at least one character, write a piece of code that assigns to an integer variable \cc{result}, the number of digits in the String \cc{str}. The expression to check whether a character \cc{ch} is a digit or not is as follows,

\begin{lstlisting}[basicstyle=\large]
boolean isDigit = false;
if(ch >= '0' && ch <= '9') {
	isDigit = true;
}
\end{lstlisting}

\begin{solution}
\begin{lstlisting}
int result = 0;
for(int i=0; i < str.length(); i++) {
	char ch = str.charAt(i);
	if(ch >= '0' && ch <= '9') {
		result++;
	}
} 
\end{lstlisting}	
\end{solution}

\question Assuming String \cc{str} contains at least one character, write a piece of code that assigns to an integer variable \cc{result}, 

\begin{itemize}
\item \cc{true}, if \cc{str} contains any spaces.	
\item \cc{false}, otherwise
\end{itemize}

\begin{solution}
\begin{lstlisting}
boolean result = false; //assume no space
for(int i=0; i < str.length(); i++) {
	char ch = str.charAt(i);
	if(ch == ' ') {
		result = true;
	}
}
\end{lstlisting}	
More efficient solution - as always in the case of \emph{validation} algorithms, no need to check once a space has been found, so, 

\begin{lstlisting}
boolean result = false; //assume no space
for(int i=0; result == false && i < str.length(); i++) {
	char ch = str.charAt(i);
	if(ch == ' ') {
		result = true;
	}
}
\end{lstlisting}	
\end{solution}

\question Assuming String \cc{str} contains at least one character, write a piece of code that assigns to an integer variable \cc{result}, 

\begin{itemize}
\item \cc{true}, if \cc{str} contains \textbf{only} spaces.	
\item \cc{false}, otherwise
\end{itemize}

\begin{solution}
\begin{lstlisting}
boolean result = true; //assume only spaces
for(int i=0; i < str.length(); i++) {
	char ch = str.charAt(i);
	if(ch != ' ') {
		result = false;
	}
}
\end{lstlisting}	
More efficient solution - as always in the case of \emph{violation} algorithms, no need to check once a space has been found, so, 

\begin{lstlisting}
boolean result = true; //assume only spaces
for(int i=0; result == true && i < str.length(); i++) {
	char ch = str.charAt(i);
	if(ch != ' ') {
		result = false;
	}
}
\end{lstlisting}	
\end{solution}
\end{questions}