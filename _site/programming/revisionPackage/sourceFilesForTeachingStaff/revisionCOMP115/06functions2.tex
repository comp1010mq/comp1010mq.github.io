\begin{questions}
\question Consider the following definition for function \cc{foo}.

\begin{lstlisting}[basicstyle=\large]
int foo(int[] a) {
	int c = 0;
	for(int i = 0; i < a.length; i++) {
		if(a[i] == 0) {
			c++;
		}
	}
	return c;	
}
\end{lstlisting}

What is the value of \cc{result} when the following code is executed?

\begin{lstlisting}[basicstyle=\large]
	int[] arr = {6,0,9,0,0,8,0,8,9,7,0,0};
	int result = foo(arr);
\end{lstlisting}

\begin{solution}
\begin{lstlisting}
6	
\end{lstlisting}	
\end{solution}
\question Consider the following definition for function \cc{foo}.

\begin{lstlisting}[basicstyle=\large]
boolean foo(int[] a, int target) {
	for(int i = 0; i < a.length; i++) {
		if(a[i] == target) {
			return true;
		}
	}
	return false;	
}
\end{lstlisting}

\begin{parts}
\part What is the value of \cc{result} when the following code is executed?

\begin{lstlisting}[basicstyle=\large]
	int[] arr = {2,1,3,6,4,9,7,8};
	boolean result = foo(arr, 5);
\end{lstlisting}

\begin{solution}
\texttt{false}
\end{solution}

\part What is the value of \cc{result} when the following code is executed?

\begin{lstlisting}[basicstyle=\large]
	int[] arr = {2,1,3,6,4,9,7,8};
	boolean result = foo(arr, 9);
\end{lstlisting}
\end{parts}

\begin{solution}
\texttt{true}
\end{solution}

\question Consider the following definition for function \cc{foo}.

\begin{lstlisting}[basicstyle=\large]
boolean foo(int[] a) {
	for(int i = 0; i < a.length; i++) {
		if(a[i]%2 == 1) {
			return false;
		}
	}
	return true;	
}
\end{lstlisting}

Write a piece of code that calls the function \cc{foo} by passing an integer array \cc{arr} and storing the value returned in a variable \texttt{result}. You may assume the array \cc{arr} has at least one item in it.

\begin{solution}
\begin{lstlisting}
boolean result = foo(arr);
\end{lstlisting}	
\end{solution}

\question Consider the following definition for function \cc{foo}.

\begin{lstlisting}[basicstyle=\large]
boolean foo(int[] a, int [] b) {
	if(a.length != b.length) {
		return false;
	}
	for(int i = 0; i < a.length; i++) {
		if(a[i] != b[i]) {
			return false;
		}
	}
	return true;	
}
\end{lstlisting}

\begin{parts}
\part Write a piece of code that calls the function \cc{foo} by passing two integer arrays \cc{myArray, yourArray} and storing the value returned in a variable \texttt{result}. You may assume the arrays \cc{myArray, yourArray} both have at least one item each.

\begin{solution}
\begin{lstlisting}
boolean result = foo(myArray, yourArray);
\end{lstlisting}	
\end{solution}

\part Explain briefly what is the purpose of function \cc{foo}.
\begin{solution}
return \texttt{true} if both arrays are identical (in terms of content), \texttt{false} otherwise.
\end{solution}
\end{parts}

\question Write a function that when passed an array containing floating-point values, doubles each item of the array. Remember that modifying the formal array inside a function modifies the actual array passed.

\begin{solution}
\begin{lstlisting}
void doubleUp(float[] arr) {
	for(int i=0; i < arr.length; i++) {
		arr[i]*=2;
	}
}
\end{lstlisting}	
\end{solution}

\question Write a function that when passed an array containing floating-point values, negates each negative item of the array so that it becomes positive.

\begin{solution}
\begin{lstlisting}
void makeAbsolute(float[] arr) {
	for(int i=0; i < arr.length; i++) {
		if(arr[i] < 0) {
			arr[i]*=-1;
		}
	}
}
\end{lstlisting}	
\end{solution}

\question Write a function that when passed an integer array, divides every even number in the array by 2.

\begin{solution}
\begin{lstlisting}
void divideEvensByTwo(float[] arr) {
	for(int i=0; i < arr.length; i++) {
		if(arr[i] % 2 == 0) {
			arr[i]/=2;
		}
	}
}	
\end{lstlisting}	
\end{solution}

\question \textbf{(somewhat challenging for most)} Write a function that when passed an integer array, resets any composite (non-prime) number to 0. You may define a \emph{helper} method.

\begin{solution}
\begin{lstlisting}
boolean isPrime(int n) { //helper
	if(n < 2) {
		return false;
	}
	for(int i=2; i*i <= n; i++) {
		if(n%i == 0) {
			return false;
		}	
	}
	return true;
}

void resetComposites(int[] arr) {
	for(int i=0; i < arr.length; i++) {
		if(isPrime(arr[i]) == false) {
			arr[i] = 0;
		}
	}
}	
\end{lstlisting}	
\end{solution}

\question \textbf{(somewhat challenging for most)} Write a function that when passed an integer array, returns an array with only the positive items from that array.

\begin{solution}
\begin{lstlisting}
int[] getPositives(int[] arr) {
	int countPositives = 0;
	for(int i=0; i < arr.length; i++) {
		if(arr[i] > 0) {
			countPositives++;
		}
	}
	
	int[] result = new int[countPositives];
	int targetIndex = 0;
	
	for(int i=0; i < arr.length; i++) {
		if(arr[i] > 0) {
			result[targetIndex] = arr[i];
			targetIndex++;
		}
	}
	
	return result;
}	
\end{lstlisting}	
\end{solution}

\end{questions}