\usepackage[T1]{fontenc}
\usepackage{pslatex}
 \usepackage[pdftex]{color}  
 \usepackage[pdftex]{graphicx}     
\usepackage{verbatim}
\usepackage{xcolor}
\usepackage{paralist}
\usepackage{tagging}

\usepackage[colorlinks=true,urlcolor=red]{hyperref}
\setlength{\topmargin}{-0.5in}                  % topmargin now at 1in
\setlength{\textheight}{9.5in}                  % body of text = 9.5in
\setlength{\oddsidemargin}{0in}                 % left margin = 1.0in on odd-numbered pages
\setlength{\evensidemargin}{0in}                % left margin = 1.0in on even-numbered pages 
\setlength{\textwidth}{6.5in}                   % width of text line.
\setlength{\parindent}{0.0in}
\newcommand{\code}{\texttt}

\usepackage{listings}
\lstset{%
	language=Java,
	basicstyle=\footnotesize\ttfamily,
	numbers=left,
	numberstyle=\tiny,        
	xleftmargin=17pt,
        	xrightmargin=5pt,
	frame=single,
	breaklines=true,
	moredelim=\item \item [is][\color{red}]{@}{@}
}

\begin{document}

\definecolor{aquamarine}{rgb}{0,0,0.7}
\definecolor{blue}{rgb}{0,0,0.7}
\definecolor{red}{rgb}{1,0,0}

%
\vspace{0.2in}
\begin{center}
        {\large  %MACQUARIE UNIVERSITY\\
%\medskip
\includegraphics[scale=0.3]{../../logo.png}\\
\medskip
        {\it  Faculty of Science and Engineering\\}
        \vspace{0.2in}
         {\bf COMP125 Fundamentals of Computer Science\\
        Workshop Week 12\\}}
\end{center}
\vspace{0.3in}
%

%\renewcommand{\labelenumi}{\arabic{enumi}.}
\renewcommand{\labelenumi}{\alph{enumi}.}
 
\section*{Learning outcomes}
This week we look at stacks. By the end of this session, you should 
\begin{enumerate}
\item have a better understanding of how stacks operate;
\item be able to use stacks to solve simple problems.
\end{enumerate}
\vskip 0.5cm

Import the java project from the archive file \texttt{stacks\_workshop\_start.zip}

\section*{Questions}

\begin{questions}

\question 
\begin{enumerate}
\item Using pen and paper, add strings "Are", "you", "my", "mummy?" to a stack in the given order.
What message would be displayed when removing elements from the stack?
\item Implement a) in Java. Namely, create a stack of {\tt String},  add items "Are", "you", "my", "mummy?" to
  the stack and use a loop to display all the elements in the stack, from top to bottom, separated by a space.
\end{enumerate}
\begin{solution}
\begin{lstlisting}
Stack<String> stk = new Stack<String>();
Stack<String> stk2 = new Stack<String>();
stk.push("Are");
stk.push("you");
stk.push("my");
stk.push("mummy?");
while(!stk.isEmpty()) {
	System.out.print(stk.peek()+" ");
	stk2.push(stk.pop());
}
while(!stk2.isEmpty()) {
	stk.push(stk2.pop());
}
System.out.println();
\end{lstlisting}
\end{solution}

\question 
\begin{enumerate}
\item How would you display the elements of a stack in the order they are entered in the stack, i.e. from
  bottom to top?
\item The class {\tt Card} deals with playing cards. It has methods to set and display the value of a
card. In the class {\tt Hand}, write a method {\tt makeHand} that
takes an integer {\tt n} and returns a stack of {\tt n} random cards. The method should also display the values of the card as they are generated and added to the stack.
\item Write a method {\tt reverseDisplay} that displays the cards in the order they are entered in the stack.
\end{enumerate}

\begin{solution}
We can use another stack or adopt a recursive approach to display the items stored in a stack in the order they were entered.

To write {\tt makeHand}, we make use of the methods {\tt setRandom} and {\tt display}. The cards are added to
the stack using {\tt push}.

See \texttt{stacks\_workshop\_solution.zip} for an implementation
\begin{lstlisting}
public static Stack<Card> makeHand(int n) {
   Stack<Card> stack = new Stack<Card>();
   for(int i = 1; i <= n; i++) {
      Card card = new Card();
      card.setRandom();
      System.out.println(card);
      stack.push(card);
   }
   System.out.println();
   return stack;
}

public static void reverseDisplay(Stack<Card> s) {
   Stack<Card> s2 = new Stack<Card>();
   while(!s.empty()) {
      s2.add(s.pop());      
   }
   while(!s2.empty()) {
      s2.pop().display();        
   }
   
}
   
public static void reverseRecDisplay(Stack<Card> s) {
   if(!s.empty()) {
      Card card = s.pop();  //the card is displayed after the call 
                            //but has been saved before    
      reverseRecDisplay(s);
      card.display();
   }           
}
\end{lstlisting}
\end{solution}


\question  \textbf{(assessed)} In class \texttt{StackDemo}, complete the method \texttt{countEven} that when passed a stack of integers, returns the number of even numbers. The stack must remain unchanged after the method finishes execution. Return \texttt{null} if the stack is empty.
\begin{solution}
\begin{lstlisting}
	public static int countEven(Stack<Integer> stk) {
		if(stk == null)
			return 0;
		
		int result = 0;
		for(Integer item: stk){
			if(item % 2 == 0) {
				result++;
			}
		}
		return result;
	}
\end{lstlisting}
\end{solution}

\question  \textbf{(assessed)} In class \texttt{StackDemo}, complete the method \texttt{alternateItems} that when passed a stack of integers, returns a stack containing every alternate item, starting from the top item, from the passed stack. For example, if the stack passed is [6, 3, 1, 8] (where 8 is the top item), returns the stack [3, 8], and when the stack passed is [2, 6, 3, 1, 8], returns the stack [2, 3, 8].

 \begin{solution}
 
 \begin{lstlisting}
 public static Stack<Integer> alternateItems(Stack<Integer> stk) {
	if(stk == null)
		return null;

	Stack<Integer> temp = new Stack<Integer>();

	while(!stk.isEmpty()) {
		temp.push(stk.pop());
		if(!stk.isEmpty())
			stk.pop();
	}

	Stack<Integer> result = new Stack<Integer>();

	while(temp.isEmpty() == false) {
		result.push(temp.pop());
	}

	return result;
	}
 \end{lstlisting}
 \end{solution}

\question \textbf{(assessed)} Write a method that when passed an array of integers, returns the array reversed, using a stack.

\begin{solution}
\begin{lstlisting}
public static int[] reverse(int[] a) {
	Stack<Integer> stk = new Stack<Integer>();
	for(int item: a)
		stk.push(item);
	int[] result = new int[a];
	int i = 0;
	while(!stk.isEmpty()) {
		result[i++] = stk.pop();
	}
	return result;
}	
\end{lstlisting}	
\end{solution}

\newpage

\section*{Revision questions}

\question Write a method that when passed an array of integers, returns the sum of all items

\question Write a method that when passed an array of integers, returns the average of all items

\question Write a method that when passed an array of \texttt{double} values, returns \texttt{true} if it is sorted in ascending order (such that each item is more than or equal to the previous item), and \texttt{false} otherwise.

\question Write a method that when passed two arrays of \texttt{double} values, returns \texttt{true} if they are exactly the same - item for item, and \texttt{false} otherwise.

\question Write a method that when passed two arrays of \texttt{double} values, returns \texttt{true} if they are mutually reverse, and \texttt{false} otherwise.

\question Write a method that when passed an array of \texttt{double} values, returns \texttt{true} if all items are positive, and \texttt{false} otherwise.

\question Write a method that when passed an integer array, returns \texttt{true} if all items are even, and \texttt{false} otherwise.

\question Write a method that when passed an array of \texttt{double} values, returns \texttt{true} if at least one item is positive, and \texttt{false} otherwise.

\question Write a method that when passed an integer array, returns \texttt{true} if at least one item is even, and \texttt{false} otherwise.

\question Write a method that when passed an integer array, returns \texttt{true} if the array follows a fibonacci sequence, that is each item of the array is the sum of the previous two items (with the first two items acting as the \emph{seed}), and \texttt{false} otherwise.

\question Write a method that when passed an array of \texttt{double} values, returns \texttt{true} if no two items of the array are the same, and \texttt{false} otherwise.

\question Perform all questions from 6 to 16, using the following data structures instead of arrays,

\begin{itemize}
\item linked list
\item array list	
\item stack
\end{itemize}

\question What is the value of \texttt{result} when the following java class is executed?

\begin{lstlisting}
public class Rec {
	public static int foo(int n) {
		if(n == 0)
			return 0;
		return n%10 + foo(n/10);
	}
	
	public static void main(String[] args) {
		int result = foo(32768);
	}
}
\end{lstlisting}

\question Write a recursive method, that when passed an integer $n$ (assume $n \geq 0$), computes and returns the sum of the first $n$ positive integers.

\question Write a recursive method, that when passed an integer $n$ (assume $n \geq 0$), computes and returns the product of the first $n$ positive integers.

\question Write a recursive method, that when passed a double $x$ an integer $n$ (assume $n \geq 0$), computes and returns $x^n$.

\newpage

\question Write definition for the following classes including data members  or instance variables (limit them to a maximum of 3) with appropriate data types, getters, setters with appropriate validation, default constructor, one or more paramterized constructors, and the following instance methods - equals(XYZ obj):boolean, compareTo(XYZ obj):int (where obj is an object of the same class XYZ as the calling object), toString():String. 

\begin{enumerate}
	\item Circle
	\item Rectangle
	\item Square
	\item Person
	\item Cube
	\item Sphere
\end{enumerate}

\question \textbf{(Pretty advanced)} Write a method that returns a 9 x 9 valid sudoku puzzle (that has at least one solution), partially filled. The design of your code should be such that the puzzle numbers and the partial filling must be \emph{clever}.
\end{questions}

\end{document}
