\begin{questions}

\begin{solution}
	
\end{solution}

\question
\begin{parts}
\part What is the value of \cc{result} when the following code is executed?

\begin{lstlisting}[basicstyle=\large]
	int result = 0;
	int a = 5;
	int b = 3;
	for(int i = 1; i <= a; i++) {
		result+=b;
	}
\end{lstlisting}

\begin{solution}
15
\end{solution}


\part What is the value stored in \cc{result} in terms of \cc{a, b}? 

\begin{solution}
\texttt{a * b}
\end{solution}
\end{parts}

\question
\begin{parts}
\part What is the value of \cc{result} when the following code is executed?

\begin{lstlisting}[basicstyle=\large]
	int result = 0;
	int a = 5;
	int b = 3;
	for(int i = 1; i < a; i++) {
		result+=b;
	}
\end{lstlisting}

\begin{solution}
12
\end{solution}

\part What is the value stored in \cc{result} in terms of \cc{a, b}? 

\begin{solution}
\texttt{(a-1) * b}		
\end{solution}
\end{parts}

\question
\begin{parts}
\part What is the value of \cc{result} when the following code is executed?

\begin{lstlisting}[basicstyle=\large]
	int result = 1;
	int x = 5;
	int n = 3;
	for(int i = 1; i <= n; i++) {
		result*=x;
	}
\end{lstlisting}

\begin{solution}
125	
\end{solution}

\part What is the value stored in \cc{result} in terms of \cc{x, n}? 

\begin{solution}
$x^n$	
\end{solution}
\end{parts}


\question What is the value of \cc{result} when the following code is executed?

\begin{lstlisting}[basicstyle=\large]
int result = 0;
for(int i = 1; i <= 10; i++) {
	if(i != 4) {
		result++;
	}
	else {
		result+=2;
	}
}
\end{lstlisting}

\begin{solution}
11
\end{solution}

\question
\begin{parts}
\part What is the value of \cc{result} when the following code is executed?

\begin{lstlisting}[basicstyle=\large]
int result = 0;
int a = 4;
int b = 3;
for(int i = 1; i <= a; i++) {
	for(int k=1; k <= b; k++) {
		result++;
	}
}
\end{lstlisting}

\begin{solution}
12
\end{solution}

\part What is the value stored in \cc{result} in terms of \cc{a, b}? 

\begin{solution}
a * b
\end{solution}

\part Based on your answer to part (b), what is the value of \cc{result} when the following code is executed?

\begin{lstlisting}[basicstyle=\large]
int result = 0;
int a = 10;
int b = 10;
for(int i = 1; i <= a; i++) {
	for(int k=1; k <= b; k++) {
		result++;
	}
}
\end{lstlisting}

\begin{solution}
100	
\end{solution}

\end{parts}

\question
\begin{parts}
\part What is the value of \cc{result} when the following code is executed?

\begin{lstlisting}[basicstyle=\large, style=buggy]
int result = 0;
int a = 10;
int b = 10;
for(int i = 1; i <= a; i++) {
	for(int k=@i@; k <= b; k++) {
		result++;
	}
}
\end{lstlisting}

\begin{solution}
55
\end{solution}

\part \textbf{(challenging)} What is the value stored in \cc{result} in terms of \cc{a, b}? 

\begin{solution}
$b + (b - 1) + \ldots + (b - a + 1)$
\end{solution}

\end{parts}

\question Write a piece of code that outputs the following in the console, using a \textbf{loop}. Use \cc{print()} statement.

\begin{verbatim}
	2 5 8 11 14 17 20 23 26 29 32 35 38 41 44 47 50
\end{verbatim}

\begin{solution}
\begin{lstlisting}
for(int i=2; i <= 50; i+=3)
	print(i+" ");	
\end{lstlisting}
\end{solution}

\question Write a piece of code that outputs the following in the console, using a \textbf{loop}. Use \cc{print()} statement.

\begin{verbatim}
	100 95 90 85 80 75 70 65 60 55 50 45 40 35 30 25 20
\end{verbatim}

\begin{solution}
\begin{lstlisting}
for(int i=100; i >= 20; i-=5)
	print(i+" ");	
\end{lstlisting}
\end{solution}

\question Write a piece of code that outputs the following in the console, using a \textbf{loop}. Use \cc{print()} statement.

\begin{verbatim}
	1 2 4 8 16 32 64 128 256 512 1024 2048 4096 8192
\end{verbatim}

\begin{solution}
\begin{lstlisting}
for(int i=1; i <= 8192; i*=2)
	print(i+" ");	
\end{lstlisting}
\end{solution}

\question Write a piece of code that outputs the following in the console, using a \textbf{loop}. Use \cc{print()} statement. \color{red}IMPORTANT: \color{black} Pay attention to the pattern, this is a tricky one.

\begin{verbatim}
	7 14 21 28 35 42 49 56 63 77 84 91 98 105 112 119 126 133 147
\end{verbatim}

\begin{solution}
\begin{lstlisting}
for(int i=7; i <= 147; i+=7) {
	if(i%10 != 0) {
		print(i+" ");
	}
}
\end{lstlisting}
\end{solution}


\question Assuming an integer variable \cc{n} such that $1 \leq n \leq 10$, write a piece of code that stores into an integer variable \texttt{fact} the product of the first \cc{n} positive integers, that is, $1 * 2 * \ldots n$.

\begin{solution}
\begin{lstlisting}
int fact = 1;
for(int i=1; i <= n; i++) {
	fact*=i;	
}
\end{lstlisting}
\end{solution}

\question Assuming an integer variable $n \geq 1$, write a piece of code that stores into a \cc{boolean} variable \texttt{isPrime},
	
	\begin{itemize}
	\item \cc{true} if \cc{n} is a prime number.
	\item \cc{false} if \cc{n} is not a prime number.	
	\end{itemize}
	
	An integer is called a \emph{prime} if it more than 1 and is divisible only by 1 and itself.
	
\begin{solution}
\begin{lstlisting}
boolean isPrime = true; //assume it's a prime
for(int i=2; i*i <= n; i++) {
	if(n%i == 0) {
		isPrime = false;
	}
}
\end{lstlisting}
\end{solution}
\end{questions}