\def\topic{Linked Lists}
\input{comp125lectureHeader}
 
\section{So why are we learning this?}

We have mentioned in earlier sections that \texttt{ArrayLists} generally offer better performance than \texttt{LinkedLists}. So why do we need to learn about \texttt{LinkedLists}? Because:

\begin{enumerate}
\item \texttt{LinkedLists} require $n$ small pockets of memory to hold $n$ values rather than 1 large block of memory.
\item \texttt{LinkedLists} are a fantastic introduction to \textit{recursive data structures}. We can extend these to binary trees, trees, and graphs.
\end{enumerate}

\section{The \texttt{Node} class}

A \texttt{Node} is a class that has two instance variables: \texttt{next: Node} and \texttt{data: <dataType>}. 

A \texttt{Node} holding \texttt{int} data is:

\begin{lstlisting}
class Node {
	public int data;
	public Node next; //reference to next node
}
\end{lstlisting}

Code to create object \texttt{p} of class \texttt{Node} and resulting memory diagram are:

\begin{lstlisting}
Node p = new Node();
p.data = 25;
p.next = null;
\end{lstlisting}


\begin{center}
\bgroup \tikzset{png export} \begin{tikzpicture}[scale=.5]
%\memoryblock{x=0,10,3,3,array={"p"}}
\memoryblock{x=0, y=10, width=3, height=3, array={"p"}, colour = green!10!white}
\memoryblock{x=5,y=9,width=6,height=5,array={"data = 25", "next = null"}, colour = red!10!white}
\draw [line width = 0.5mm, -{Latex[length=5mm]}] (2.5, 11.5) -- (5, 11.5);
\end{tikzpicture} \egroup
\end{center}

Code to create objects \texttt{p, q} of class \texttt{Node} and resulting memory diagram are:

\begin{lstlisting}
Node q = new Node();
q.data = 42;
q.next = null;
Node p = new Node();
p.data = 25;
p.next = q;
\end{lstlisting}

\begin{center}
\bgroup \tikzset{png export} \begin{tikzpicture}[scale=.5]
\memoryblock{x=0,y=10,width=3,height=3,array={"p"}, colour = green!10!white}
\memoryblock{x=5,y=9,width=5,height=5,array={"data = 25", "next"}, , colour = red!10!white}
\memoryblock{x=11,y=11,width=3,height=3,array={"q"}, colour = green!10!white}
\memoryblock{x=16,y=8,width=5,height=5,array={"data = 42", "next = null"}, colour=red!10!white}
\draw [line width = 0.5mm, -{Latex[length=5mm]}] (2.5, 11.5) -- (5, 11.5);
\draw [line width = 0.5mm, -{Latex[length=5mm]}] (8.8, 10) -- (16, 10);
\draw [line width = 0.5mm, -{Latex[length=5mm]}] (13.2, 12) -- (16, 10.2);
\end{tikzpicture} \egroup
\end{center}

Here, we can see that \texttt{p.next} (of type \texttt{Node}) is a shallow copy of \texttt{q} (of type \texttt{Node}).

The \texttt{data} instance variable in a \texttt{Node} class may be object of another class too. For example,

\begin{lstlisting}
class Rectangle {
	public double width, height;
}  
\end{lstlisting}

\begin{lstlisting}
class Node {
	public Rectangle data;
	public Node next;
}
\end{lstlisting}

\begin{lstlisting}
Rectangle r1 = new Rectangle();
r1.width = 2.5;
r1.height = 1.5;
Rectangle r2 = new Rectangle();
r2.width = 4.2;
r2.height = 3.6;
Node q = new Node();
q.data = r2;
q.next = null;
Node p = new Node();
p.data = r1;
p.next = q;
\end{lstlisting}

The memory diagram for this code is below. Notice they key points:

\begin{itemize}
\item \texttt{p.next (Node)} is a shallow copy of \texttt{q (Node)}
\item \texttt{p.data (Rectangle)} is a shallow copy of \texttt{r1 (Rectangle)}
\item \texttt{q.data (Rectangle)} is a shallow copy of \texttt{r2 (Rectangle)}
\end{itemize}

\begin{center}
\bgroup \tikzset{png export} \begin{tikzpicture}[scale=.5]
\memoryblock{x=0,y=8,width=2,height=2,array={"p"}, colour = green!10!white}
\memoryblock{x=4,y=6.5,width=4,height=5,array={"data", "next"}, colour=red!10!white}
\memoryblock{x=6,y=12,width=2,height=2,array={"r1"}, colour = green!10!white}
\memoryblock{x=11,y=9,width=5,height=5,array={"width=2.5", "height=1.5"}, colour=red!10!white	}
\draw [line width = 0.5mm, -{Latex[length=5mm]}] (7.5, 13) -- (11, 13);
\draw [line width = 0.5mm, -{Latex[length=5mm]}] (7.5, 10) -- (11, 10);
\draw [line width = 0.5mm, -{Latex[length=5mm]}] (1.5, 9) -- (4, 9);
\draw [line width = 0.5mm, -{Latex[length=5mm]}] (7.5, 8) -- (16, 8);
\memoryblock{x=11,y=5,width=2,height=2,array={"q"}, colour = green!10!white}
\memoryblock{x=16,y=4,width=4,height=5,array={"data", "next"}, colour=red!10!white}
\draw [line width = 0.5mm, -{Latex[length=5mm]}] (12.5, 6) -- (16, 6);
\memoryblock{x=21,y=9.5,width=2,height=2,array={"r2"}, colour = green!10!white}
\memoryblock{x=25,y=6.5,width=5,height=5,array={"width=4.2", "height=3.6"}, colour=red!10!white}
\draw [line width = 0.5mm, -{Latex[length=5mm]}] (19.5, 7.5) -- (25, 7.5);
\draw [line width = 0.5mm, -{Latex[length=5mm]}] (22.5, 10.5) -- (25, 10.5);
\end{tikzpicture} \egroup
\end{center}

\begin{exercise}[5]
Draw a memory diagram for the following code:

\begin{lstlisting}[basicstyle=\small]
class Circle {
	private double radius;
	public Circle(double r) { 
		radius = Math.abs(r); 
	}
}
class Node {
	private Circle data;
	private Node next;
	public Node(Circle d, Node n) { 
		data = d; 
		next = n; 
	}
}
public class Client {
	public static void main(String[] args) {
		Circle c1 = new Circle(2.8);
		Circle c2 = new Circle(1.6);
		Node p = new Node(c1, null);
		Node q = new Node(c2, p);
		Node r = new Node(c1, p);
	}
}
\end{lstlisting}	
\end{exercise}
\begin{answer}
Solution not provided for this exercise	
\end{answer}

\section{Creating a \emph{"linked list"} manually}

Once we understand the \texttt{Node} class, we can link objects of \texttt{Node} together to create a \textit{"linked" list}.

\begin{lstlisting}
class Node {
	private int data; //primitive data type for simplicity
	private Node next;
	
	//getter, setter for data
	public int getData() { return data; }	
	public void setData(int d) { data = d; }
	
	//getter, setter for next
	public Node getNext() {	return next; }
	public void setNext(Node n) { next = n; }
	
	public Node(int d) { 
		data = d; 
		next = null; 
	}
	public Node(int d, Node n) { 
		data = d; 
		next = n; 
	}
}
\end{lstlisting}

\newpage

\begin{lstlisting}
class Client {
	public static void main(String[] args) {
		Node n5 = new Node(20, null);
		Node n4 = new Node(70, n5);
		Node n3 = new Node(10, n4);
		Node n2 = new Node(90, n3);
		Node n1 = new Node(30, n2);
	}
}
\end{lstlisting}

The memory diagram for the above code is given below:

\vskip 0.5cm

\begin{center}
\bgroup \tikzset{png export} \begin{tikzpicture}[scale=.5]
\memoryblock{x=1.5,y=10,width=2,height=2,array={"n1"}, colour=green!10!white}
\memoryblock{x=7.5,y=10,width=2,height=2,array={"n2"}, colour=green!10!white}
\memoryblock{x=13.5,y=10,width=2,height=2,array={"n3"}, colour=green!10!white}
\memoryblock{x=19.5,y=10,width=2,height=2,array={"n4"}, colour=green!10!white}
\memoryblock{x=25.5,y=10,width=2,height=2,array={"n5"}, colour=green!10!white}

\memoryblock{x=0,y=5,width=5,height=3,array={"data=30", "next"}, colour=red!10!white}
\memoryblock{x=6,y=5,width=5,height=3,array={"data=90", "next"}, colour=red!10!white}
\memoryblock{x=12,y=5,width=5,height=3,array={"data=10", "next"}, colour=red!10!white}
\memoryblock{x=18,y=5,width=5,height=3,array={"data=70", "next"}, colour=red!10!white}
\memoryblock{x=24,y=5,width=5,height=3,array={"data=20", "next=null"}, colour=red!10!white}
\draw [line width = 0.5mm, -{Latex[length=3mm]}] (4, 6) -- (6, 6);
\draw [line width = 0.5mm, -{Latex[length=3mm]}] (10, 6) -- (12, 6);
\draw [line width = 0.5mm, -{Latex[length=3mm]}] (16, 6) -- (18, 6);
\draw [line width = 0.5mm, -{Latex[length=3mm]}] (22, 6) -- (24, 6);

\draw [line width = 0.5mm, -{Latex[length=3mm]}] (2.5, 10.5) -- (2.5, 8);
\draw [line width = 0.5mm, -{Latex[length=3mm]}] (8.5, 10.5) -- (8.5, 8);
\draw [line width = 0.5mm, -{Latex[length=3mm]}] (14.5, 10.5) -- (14.5, 8);
\draw [line width = 0.5mm, -{Latex[length=3mm]}] (20.5, 10.5) -- (20.5, 8);
\draw [line width = 0.5mm, -{Latex[length=3mm]}] (26.5, 10.5) -- (26.5, 8);
\end{tikzpicture} \egroup
\end{center}

\subsection{Simplified representation of linked nodes}

The above diagram, while thorough, is \emph{too} detailed. We apply a little abstraction and represent the same diagram as:

\begin{center}
\bgroup \tikzset{png export} \begin{tikzpicture}
\foreach \i in {0,...,4} {
	\def\data{{30, 90, 10, 70, 20}}
	\filldraw [fill=white!80!red, rounded corners = 8pt, line width = 0.25mm] (3*\i, 0) rectangle (2+3*\i, 1);
	\draw (1+3*\i,0) -- (1+3*\i,1);
	\pgfmathparse{1+\i}
	\edef\number{\pgfmathresult}
	\node at (1+3*\i,-0.5) {\large n\pgfmathprintnumber[assume math mode=true]{\number} \normalsize}; 
	\pgfmathparse{\data[\i]}
	\edef\item{\pgfmathresult}

	\node at (0.5+3*\i,0.5) {\large \item \normalsize}; 
	\draw [line width = 0.5mm, -{Latex[length=2mm]}] (1.5+3*\i, 0.5) -- (3+3*\i, 0.5);	
}
\node at (15.5, 0.5) {null};
\end{tikzpicture} \egroup
\end{center}

\newpage

\subsection{Traversing a linked list}

\subsubsection{Incorrect Approach}

\begin{lstlisting}
while(n1 != null) {
	n1 = n1.getNext();
}
\end{lstlisting}

You are modifying the reference \texttt{n1}. Each time you update \texttt{n1}, the instance referred by \texttt{n1} before the update is deleted from memory. Thereby, all the nodes will be erased from the memory at the end of execution. So yeah, not a good idea.
\vskip 0.5cm
\bgroup \tikzset{png export} \begin{tikzpicture}[scale=0.9]
\foreach \i in {0,...,4} {
	\def\data{{30, 90, 10, 70, 20}}
	\def\labels{{"n1","n2","n3","n4","n5"}}
	\filldraw [fill=white!80!green, rounded corners = 8pt, line width = 0.25mm] (3*\i, 0) rectangle (2+3*\i, 1);
	\draw (1+3*\i,0) -- (1+3*\i,1);
	\pgfmathparse{1+\i}
	\edef\number{\pgfmathresult}
	
	\pgfmathparse{\labels[\i]}
	\edef\label{\pgfmathresult}
	\ifthenelse{\i=0}
	{
		\node at (1+3*\i,-0.6) {\textbf{\label}}; 
	}
	{
		\node at (1+3*\i,-0.6) {\label}; 
	}
	\pgfmathparse{\data[\i]}
	\edef\item{\pgfmathresult}
	\node at (0.5+3*\i,0.5) {\text{\small \item}}; 
	
	\draw [line width = 0.5mm, -{Latex[length=2mm]}] (1.5+3*\i, 0.5) -- (3+3*\i, 0.5);	
}
\node at (15.5, 0.5) {null};
\end{tikzpicture} \egroup
\vskip 0.3cm

\bgroup \tikzset{png export} \begin{tikzpicture}[scale=0.9]
\foreach \i in {0,...,4} {
	\def\data{{30, 90, 10, 70, 20}}
	\def\labels{{"unreferenced (deleted)", "n1, n2","n3","n4","n5"}}
	\ifthenelse{\i=0}
	{
		\filldraw [fill=white!80!red, rounded corners = 8pt, line width = 0.25mm] (3*\i, 0) rectangle (2+3*\i, 1);
	}
	{
		\filldraw [fill=white!80!green, rounded corners = 8pt, line width = 0.25mm] (3*\i, 0) rectangle (2+3*\i, 1);
	}
	\draw (1+3*\i,0) -- (1+3*\i,1);
	\pgfmathparse{1+\i}
	\edef\number{\pgfmathresult}
	
	\pgfmathparse{\labels[\i]}
	\edef\label{\pgfmathresult}
	\ifthenelse{\i=1}
	{
	\node [text width = 2cm] at (1.5+3*\i,-0.6) {\textbf{\label}}; 
	}
	{
		\ifthenelse{\i=0}
		{
			\node [text width = 2cm] at (1+3*\i,-0.6) {\label}; 
		}
		{
			\node [text width = 2cm] at (2+3*\i,-0.6) {\label}; 
		}
	}
	\pgfmathparse{\data[\i]}
	\edef\item{\pgfmathresult}
	\node at (0.5+3*\i,0.5) {\text{\small \item}}; 
	
	\draw [line width = 0.5mm, -{Latex[length=2mm]}] (1.5+3*\i, 0.5) -- (3+3*\i, 0.5);	
}
\node at (15.5, 0.5) {null};
\end{tikzpicture} \egroup
\vskip 0.3cm


\bgroup \tikzset{png export} \begin{tikzpicture}[scale=0.9]
\foreach \i in {0,...,3} {
	\def\data{{90, 10, 70, 20}}
	\def\labels{{"n2","n1, n3","n4","n5"}}
	\ifthenelse{\i=0}
	{
		\filldraw [fill=white!80!red, rounded corners = 8pt, line width = 0.25mm] (3*\i, 0) rectangle (2+3*\i, 1);
	}
	{
		\filldraw [fill=white!80!green, rounded corners = 8pt, line width = 0.25mm] (3*\i, 0) rectangle (2+3*\i, 1);
	}
	\draw (1+3*\i,0) -- (1+3*\i,1);
	\pgfmathparse{1+\i}
	\edef\number{\pgfmathresult}
	
	\pgfmathparse{\labels[\i]}
	\edef\label{\pgfmathresult}
	\ifthenelse{\i=1}
	{
	\node [text width = 2cm] at (1.5+3*\i,-0.6) {\textbf{\label}}; 
	}
	{
	\node [text width = 2cm] at (2+3*\i,-0.6) {\label}; 
	}
	\pgfmathparse{\data[\i]}
	\edef\item{\pgfmathresult}
	\node at (0.5+3*\i,0.5) {\text{\small \item}}; 
	
	\draw [line width = 0.5mm, -{Latex[length=2mm]}] (1.5+3*\i, 0.5) -- (3+3*\i, 0.5);	
}
\node at (12.5, 0.5) {null};
\end{tikzpicture} \egroup
\vskip 0.3cm

\bgroup \tikzset{png export} \begin{tikzpicture}[scale=0.9]
\foreach \i in {0,...,3} {
	\def\data{{90, 10, 70, 20}}
	\def\labels{{"n2","n3","n1, n4","n5"}}
	\ifthenelse{\i=0}
	{
		\filldraw [fill=white!80!red, rounded corners = 8pt, line width = 0.25mm] (3*\i, 0) rectangle (2+3*\i, 1);
	}
	{
		\filldraw [fill=white!80!green, rounded corners = 8pt, line width = 0.25mm] (3*\i, 0) rectangle (2+3*\i, 1);
	}
	\draw (1+3*\i,0) -- (1+3*\i,1);
	\pgfmathparse{1+\i}
	\edef\number{\pgfmathresult}
	
	\pgfmathparse{\labels[\i]}
	\edef\label{\pgfmathresult}
	\ifthenelse{\i=2}
	{
	\node [text width = 2cm] at (1.5+3*\i,-0.6) {\textbf{\label}}; 
	}
	{
	\node [text width = 2cm] at (2+3*\i,-0.6) {\label}; 
	}	
	\pgfmathparse{\data[\i]}
	\edef\item{\pgfmathresult}
	\node at (0.5+3*\i,0.5) {\text{\small \item}}; 
	
	\draw [line width = 0.5mm, -{Latex[length=2mm]}] (1.5+3*\i, 0.5) -- (3+3*\i, 0.5);	
}
\node at (12.5, 0.5) {null};
\end{tikzpicture} \egroup
\vskip 0.3cm

\bgroup \tikzset{png export} \begin{tikzpicture}[scale=0.9]
\foreach \i in {0,...,3} {
	\def\data{{90, 10, 70, 20}}
	\def\labels{{"n2","n3","n4","n1, n5"}}
	\ifthenelse{\i=0}
	{
		\filldraw [fill=white!80!red, rounded corners = 8pt, line width = 0.25mm] (3*\i, 0) rectangle (2+3*\i, 1);
	}
	{
		\filldraw [fill=white!80!green, rounded corners = 8pt, line width = 0.25mm] (3*\i, 0) rectangle (2+3*\i, 1);
	}
	\draw (1+3*\i,0) -- (1+3*\i,1);
	\pgfmathparse{1+\i}
	\edef\number{\pgfmathresult}
	
	\pgfmathparse{\labels[\i]}
	\edef\label{\pgfmathresult}
	\ifthenelse{\i=3}
	{
	\node [text width = 2cm] at (1.5+3*\i,-0.6) {\textbf{\label}}; 
	}
	{
	\node [text width = 2cm] at (2+3*\i,-0.6) {\label}; 
	}	
	\pgfmathparse{\data[\i]}
	\edef\item{\pgfmathresult}
	\node at (0.5+3*\i,0.5) {\text{\small \item}}; 
	
	\draw [line width = 0.5mm, -{Latex[length=2mm]}] (1.5+3*\i, 0.5) -- (3+3*\i, 0.5);	
}
\node at (12.5, 0.5) {null};
\end{tikzpicture} \egroup
\vskip 0.3cm

\bgroup \tikzset{png export} \begin{tikzpicture}[scale=0.9]
\foreach \i in {0,...,3} {
	\def\data{{90, 10, 70, 20}}
	\def\labels{{"n2","n3","n4","n5"}}
	\ifthenelse{\i=0}
	{
		\filldraw [fill=white!80!red, rounded corners = 8pt, line width = 0.25mm] (3*\i, 0) rectangle (2+3*\i, 1);
	}
	{
		\filldraw [fill=white!80!green, rounded corners = 8pt, line width = 0.25mm] (3*\i, 0) rectangle (2+3*\i, 1);
	}
	\draw (1+3*\i,0) -- (1+3*\i,1);
	\pgfmathparse{1+\i}
	\edef\number{\pgfmathresult}
	
	\pgfmathparse{\labels[\i]}
	\edef\label{\pgfmathresult}
	\node [text width = 2cm] at (2+3*\i,-0.6) {\label}; 
	
	\pgfmathparse{\data[\i]}
	\edef\item{\pgfmathresult}
	\node at (0.5+3*\i,0.5) {\text{\small \item}}; 
	
	\draw [line width = 0.5mm, -{Latex[length=2mm]}] (1.5+3*\i, 0.5) -- (3+3*\i, 0.5);	
}
\node at (12.5, 0.5) {null};
\node at (12.5, -0.5) {\textbf{n1}};
\end{tikzpicture} \egroup
\vskip 0.5cm

This results in us losing the reference to the first item (\texttt{n1}) and the object being deleted from the memory.
\newpage

\subsubsection{Correct Approach}

We shallow copy the starting node into a \textit{traversal} node. Then we shift it forward every time in the loop by shallow copying the \texttt{next} instance variable into it. 

\begin{lstlisting}
Node temp = n1;
while(temp != null) {
	temp = temp.getNext();
}
\end{lstlisting}

\vskip 0.5cm
\bgroup \tikzset{png export} \begin{tikzpicture}[scale=0.85]
\foreach \k in {-1,...,4} {
\foreach \i in {0,...,4} {
	\def\data{{30, 90, 10, 70, 20}}
	\def\labels{{"n1","n2","n3","n4","n5"}}
	\draw (1+3*\i,3*\k) -- (1+3*\i,1+3*\k);
	\pgfmathparse{1+\i}
	\edef\number{\pgfmathresult}
	
	\pgfmathparse{\labels[\i]}
	\edef\label{\pgfmathresult}
	
	\pgfmathparse{4-\k}
	\edef\m{\pgfmathresult}
	
	\ifthenelse{\i=\m}
	{
		\filldraw [fill=white!80!green, rounded corners = 8pt, line width = 0.75mm] (3*\i, 3*\k) rectangle (2+3*\i, 1+3*\k);
		\node [blue] at (1+3*\i,-0.6+3*\k) {\Large temp, \label \normalsize}; 
	}
	{
		\filldraw [fill=white!80!red, rounded corners = 8pt, line width = 0.25mm] (3*\i, 3*\k) rectangle (2+3*\i, 1+3*\k);
		\node at (1+3*\i,-0.6+3*\k) {\label}; 
	}
	
	\pgfmathparse{\data[\i]}
	\edef\item{\pgfmathresult}
	\node at (0.5+3*\i,0.5+3*\k) {\text{\small \item}}; 
	
	\draw [line width = 0.5mm, -{Latex[length=2mm]}] (1.5+3*\i, 0.5+3*\k) -- (3+3*\i, 0.5+3*\k);	
}
\node at (15.5, 0.5+3*\k) {null};
}
\node [blue] at (15.5, -3.5) {\Large temp \normalsize};
\end{tikzpicture} \egroup

\newpage

\subsection{Some examples of traversal}

\begin{enumerate}
\item Counting the number of linked nodes:
\begin{lstlisting}
Node temp = n1;
int counter = 0;
while(temp != null) {
	counter++;
	temp = temp.getNext();
}
\end{lstlisting}

\item Adding the data values in the linked nodes:
\begin{lstlisting}
Node temp = n1;
int total = 0;
while(temp != null) {
	total = total + temp.getData();
	temp = temp.getNext();
}
\end{lstlisting}

\item Adding values over 30 in the linked nodes:
\begin{lstlisting}
Node temp = n1;
int total = 0;
while(temp != null) {
	if(temp.getData() > 30) {
		total = total + temp.getData();
	}
	temp = temp.getNext();
}
\end{lstlisting}

\newpage

\item Determining highest data value in the linked nodes (assuming list is not empty):
\begin{lstlisting}
Node temp = n1;
int highest = temp.getData();
while(temp != null) {
	if(temp.getData() > highest) {
		highest = temp.getData();
	}
	temp = temp.getNext();
}
\end{lstlisting}

\item Determining if each item is different from the other:
\begin{lstlisting}[linewidth=14cm]
Node temp = n1;
boolean foundDuplicate = false; 
while(temp != null && !foundDuplicate) {
	Node temp2 = temp.getNext();
	while(temp2 != null && !foundDuplicate) {
		if(temp.getData() == temp2.getData()) {//DUPLICATE
			foundDuplicate = true; 
		}
		temp2 = temp2.getNext();
	}
	temp = temp.getNext();
}
\end{lstlisting}
\end{enumerate}

\newpage

Obviously, we can pass a \texttt{Node} object to a method just like any other object.

\begin{enumerate}
\item Example 1: Counting occurrences of an item
\begin{lstlisting}
public static int countOccurrences(Node node, int item) {
	int result = 0;
	/*
		note that node is a shallow copy of 
		the actual object passed to the method 
		call and hence can be modified without
		modifying the actual object. 
	*/
	while(node != null) { 
		if(node.getData() == item) {
			result++;
		}
		node = node.getNext();
	}
	return result;
}	
\end{lstlisting}

\item Example 2: Checking if all values are positive
\begin{lstlisting}
public static boolean allPositives(Node node) {
	while(node != null) { 
		if(node.getData() <= 0) {
			return false;
		}
		node = node.getNext();
	} //end loop
	return true;
}	
\end{lstlisting}
\end{enumerate}

\newpage

\subsubsection{Recursive methods on linked nodes}

\textbf{Advanced} A brilliant example of how this is useful is given in the following method:

\begin{lstlisting}
/*
return true the sum of some items starting at node n
is total, false otherwise
*/
public static boolean addsTo(Node node, int total) {
	if(total == 0)
		return true;	
	//now we know total is not 0
	if(node == null) 
		return false;
	int remaining = total - node.getData();
	if(addsTo(node.getNext(), remaining)) {
		/*
		there is a combination after node for 
		(total minus what node contains)
		*/
		return true;
	}
	else {
		/*
		there is no combination after node for
		(total minus what node contains) so check for a
		 combination after node for total
		*/
		return addsTo(node.getNext(), total);
	}	
}	
\end{lstlisting}

\section{Custom-built LinkedList class}

In this next and final step, we put the starting node in a class and operate on the list using the starting node.

\begin{lstlisting}[linewidth=15.5cm]
class MyLinkedList {
	private Node head;
	
	public MyLinkedList() {
		head = null;
	}
	
	/**
	insert the passed node object at beginning of list
	*/
	public void add(Node node) {
		if(head == null) { //empty list
			head = node; 
		}
		else { //not empty
			node.setNext(head); //link so head follows node
			head = node; //update  reference for starting node
		}
	}
	
	public String toString() {
		Node temp = head;
		String result = "[";
		while(temp != null) {
			result = result + temp.getData() + ", ";
			temp = temp.getNext();
		}
		result = result.substring(0, result.length()-2); 
		//remove the last ", "
		return result + "]";
	}
}
\end{lstlisting}

A linked list \texttt{myList} where head holds a reference to a node with data 30, whose \texttt{next} instance variable holds a reference to a node with data 90, whose \texttt{next} instance variable holds a reference to a node with data 10, whose \texttt{next} instance variable holds a reference to a node with data 70, whose \texttt{next} instance variable holds a reference to a node with data 20, is given below.

\vskip 0.5cm

\bgroup \tikzset{png export} \begin{tikzpicture}[scale=0.8]
\memoryblock{x=0,y=5,width=2,height=2,array={"myList"}}
\memoryblock{x=0,y=2,width=2,height=2,array={"head"}}
\draw [line width = 0.5mm, -{Latex[length=4mm]}] (1, 5) -- (1, 4);	
\draw [line width = 0.5mm, -{Latex[length=4mm]}] (1, 2) -- (1, 1);	
\foreach \i in {0,...,4} {
	\def\data{{30, 90, 10, 70, 20}}
	\filldraw [fill=white!80!red, rounded corners = 8pt, line width = 0.25mm] (3*\i, 0) rectangle (2+3*\i, 1);
	\draw (1+3*\i,0) -- (1+3*\i,1);
	\pgfmathparse{1+\i}
	\edef\number{\pgfmathresult}
	
	\pgfmathparse{\data[\i]}
	\edef\item{\pgfmathresult}
	\node at (0.5+3*\i,0.5) {\text{\small \item}}; 
	
	\draw [line width = 0.5mm, -{Latex[length=2mm]}] (1.5+3*\i, 0.5) -- (3+3*\i, 0.5);	
}
\node at (15.5, 0.5) {null};
\end{tikzpicture} \egroup

\begin{exercise}[5]
Define a method \texttt{total()} that returns the sum of data values of all nodes in a list based along the lines of \texttt{toString()} method.	
\end{exercise}
\begin{answer}
\begin{lstlisting}
public int total() {
	Node temp = head;
	int result = 0;
	while(temp != null) {
		result = result + temp.getData();
		temp = temp.getNext();
	}
	return result;
}
\end{lstlisting}	
\end{answer}

\begin{exercise}[8]
Define a method \texttt{highest()} that returns the highest value in the list (\texttt{null} if list is empty). Note that since \texttt{null} is required as error return status, return type should be \texttt{Integer}, not \texttt{int}.
\end{exercise}
\begin{answer}
\begin{lstlisting}
public Integer highest() {
	if(head == null)
		return null;
	Node temp = head;
	int result = head.getData();
	while(temp != null) {
		if(temp.getData() > result) {
			result = temp.getData();
		temp = temp.getNext();
	}
	return result;
}
\end{lstlisting}	
\end{answer}

\newpage

\subsection{Accessing an item at an arbitrary index}

Assuming that indexing begins with 0, we can write a method \texttt{get(int idx)} that returns the value of an item at an arbitrary index \texttt{idx}.

First, we should write a method \texttt{size()} since valid indices would then be \texttt{[0, ..., size()-1]}.

\begin{lstlisting}
public int size() {
	int count = 0;
	Node temp = head;
	while(temp!=null) {
		count++;
		temp = temp.getNext();
	}
	return count;
}
\end{lstlisting}

Our method \texttt{get(int idx)} is:

\begin{lstlisting}
public Integer get(int idx) {
	if(idx < 0 || idx >= size()) {
		return null;
	}
	
	Node temp = head;
	/*
	move forward idx times. 
	if idx = 0, don't move forward at all
	if idx = 4, move forward four times
	...
	*/
	for(int i=0; i < idx; i++) {
		temp = temp.getNext();
	}
	return temp.getData();
}
\end{lstlisting}

\subsection{Inserting an item at an arbitrary index}

We can either pass the item to be inserted (in our case, an integer), or a \texttt{Node} containing the item as instance variable \texttt{data} as shown below:

\vskip 0.5cm

\bgroup \tikzset{png export} \begin{tikzpicture}
	\memoryblock{colour=white!80!blue, width=1.5, height=1.5, array={"node"}}
	\memoryblock{colour=white!80!blue, x = 3, y = -0.25, width=3, height=2, array={"data=50", "next=null"}}
	\arrow{startX=1.25, startY=0.75, endX=3, endY=0.75}
\end{tikzpicture} \egroup

\vskip 0.5cm

For simplicity, the diagram is reduced as follows,

\vskip 0.5cm

\bgroup \tikzset{png export} \begin{tikzpicture}
\filldraw [fill=white!80!blue, rounded corners = 8pt, line width = 0.25mm] (0, 0) rectangle (2, 1);
\draw (1,0) -- (1,1);
\node at (1, -0.5) {node};
\node at (0.5, 0.5) {50};
\node at (1.5, 0.5) {null};
\node at (1, -0.5) {node};
\end{tikzpicture} \egroup
 
\subsubsection{SCENARIO 1: Inserting in an empty list}

We can only insert at index 0 in an empty list. When the list is empty, \texttt{head} is \texttt{null}.

\vskip 0.5cm

\bgroup \tikzset{png export} \begin{tikzpicture}
	\memoryblock{colour=white!80!red, width=2.5, height=1.5, array={"head = null"}}
\end{tikzpicture} \egroup

\vskip 0.5cm

In this case, all we have to do is re-refer \texttt{head} to \texttt{node}.

\vskip 0.5cm

\bgroup \tikzset{png export} \begin{tikzpicture}
\memoryblock{colour=white!80!red, width=1.5, height=1.5, array={"head"}}
	
\filldraw [fill=white!80!blue, rounded corners = 8pt, line width = 0.25mm] (3, 0) rectangle (6, 1.5);
\draw (4.5,0) -- (4.5,1.5);
\node at (4.5, -0.5) {node};
\node at (3.75, 0.75) {50};
\node at (5.25, 0.75) {null};
\node at (4.5, -0.5) {node};

\arrow{startX=1.5, startY=0.75, endX=3, endY=0.75}
\end{tikzpicture} \egroup

\newpage

\subsubsection{SCENARIO 2: Inserting in a non-empty list}

We can insert at any index from 0 (insert before the first item) to \texttt{size()} (insert after the last item).

If we are going to insert at index 0, \texttt{head} must be updated. In fact, this code also works when list is empty (\texttt{head} is \texttt{null}).

\begin{lstlisting}
if(index == 0) {
	node.setNext(head);
	head = node;
}
\end{lstlisting}

For any index more than 0, we follow the procedure described below:

Let's say we want to insert a node with data 50 at index 3 (after the 3rd item) in the following list.

\vskip 0.5cm

\bgroup \tikzset{png export} \begin{tikzpicture}
\node at (1,-0.5) {\text{\small head}}; 
\foreach \i in {0,...,4} {
	\def\data{{30, 90, 10, 70, 20}}
	\filldraw [fill=white!80!red, rounded corners = 8pt, line width = 0.25mm] (3*\i, 0) rectangle (2+3*\i, 1);
	\draw (1+3*\i,0) -- (1+3*\i,1);
	\pgfmathparse{1+\i}
	\edef\number{\pgfmathresult}
	
	\pgfmathparse{\data[\i]}
	\edef\item{\pgfmathresult}
	\node at (0.5+3*\i,0.5) {\text{\small \item}}; 
	
	\draw [line width = 0.5mm, -{Latex[length=2mm]}] (1.5+3*\i, 0.5) -- (3+3*\i, 0.5);	
}
\node at (15.5, 0.5) {null};
\end{tikzpicture} \egroup

We must reach the 3rd item (at index 2) and manipulate it's \texttt{next} instance variable.

Starting with \texttt{head}, how many times must we \emph{move forward} to reach the item at index 2? That's right - 2 times.

\begin{lstlisting}
Node current = head;
for(int i=1; i < 3; i++) {
	current = current.getNext();
}
\end{lstlisting}

Initial state:
\vskip 0.5cm

\bgroup \tikzset{png export} \begin{tikzpicture}
\node at (0,-0.5) {\text{\small head, }}; 	
\begin{scope}[text=red]
\node at (1.2,-0.5) {\text{\small current}}; 	
\end{scope}
\foreach \i in {0,...,4} {
	\def\data{{30, 90, 10, 70, 20}}
	\ifthenelse{\i=0} {
		\filldraw [fill=white!80!green, rounded corners = 8pt, line width = 0.25mm] (3*\i, 0) rectangle (2+3*\i, 1);
	}
	{
		\filldraw [fill=white!80!red, rounded corners = 8pt, line width = 0.25mm] (3*\i, 0) rectangle (2+3*\i, 1);
	}
	\draw (1+3*\i,0) -- (1+3*\i,1);
	\pgfmathparse{1+\i}
	\edef\number{\pgfmathresult}
	
	\pgfmathparse{\data[\i]}
	\edef\item{\pgfmathresult}
	\node at (0.5+3*\i,0.5) {\text{\small \item}}; 
	
	\draw [line width = 0.5mm, -{Latex[length=2mm]}] (1.5+3*\i, 0.5) -- (3+3*\i, 0.5);	
}
\node at (15.5, 0.5) {null};
\end{tikzpicture} \egroup

After iteration 1:
\vskip 0.5cm

\bgroup \tikzset{png export} \begin{tikzpicture}
\node at (1,-0.5) {\text{\small head}}; 
\begin{scope}[text=red]
\node at (4,-0.5) {\text{\small current}}; 	
\end{scope}
\foreach \i in {0,...,4} {
	\def\data{{30, 90, 10, 70, 20}}
	\ifthenelse{\i=1} {
		\filldraw [fill=white!80!green, rounded corners = 8pt, line width = 0.25mm] (3*\i, 0) rectangle (2+3*\i, 1);
	}
	{
		\filldraw [fill=white!80!red, rounded corners = 8pt, line width = 0.25mm] (3*\i, 0) rectangle (2+3*\i, 1);
	}
	\draw (1+3*\i,0) -- (1+3*\i,1);
	\pgfmathparse{1+\i}
	\edef\number{\pgfmathresult}
	
	\pgfmathparse{\data[\i]}
	\edef\item{\pgfmathresult}
	\node at (0.5+3*\i,0.5) {\text{\small \item}}; 
	
	\draw [line width = 0.5mm, -{Latex[length=2mm]}] (1.5+3*\i, 0.5) -- (3+3*\i, 0.5);	
}
\node at (15.5, 0.5) {null};
\end{tikzpicture} \egroup

After iteration 2:
\vskip 0.5cm

\bgroup \tikzset{png export} \begin{tikzpicture}
\node at (1,-0.5) {\text{\small head}}; 
\begin{scope}[text=red]
\node at (7,-0.5) {\text{\small current}}; 	
\end{scope}
\foreach \i in {0,...,4} {
	\def\data{{30, 90, 10, 70, 20}}
	\ifthenelse{\i=2} {
		\filldraw [fill=white!80!green, rounded corners = 8pt, line width = 0.25mm] (3*\i, 0) rectangle (2+3*\i, 1);
	}
	{
		\filldraw [fill=white!80!red, rounded corners = 8pt, line width = 0.25mm] (3*\i, 0) rectangle (2+3*\i, 1);
	}	\draw (1+3*\i,0) -- (1+3*\i,1);
	\pgfmathparse{1+\i}
	\edef\number{\pgfmathresult}
	
	\pgfmathparse{\data[\i]}
	\edef\item{\pgfmathresult}
	\node at (0.5+3*\i,0.5) {\text{\small \item}}; 
	
	\draw [line width = 0.5mm, -{Latex[length=2mm]}] (1.5+3*\i, 0.5) -- (3+3*\i, 0.5);	
}
\node at (15.5, 0.5) {null};
\end{tikzpicture} \egroup

Then we have a reference to the 3rd item in \texttt{current}.

The node after \texttt{current} should be after the node to inserted.

\begin{lstlisting}
node.setNext(current.getNext());
\end{lstlisting}

\bgroup \tikzset{png export} \begin{tikzpicture}
\node at (1,-0.5) {\text{\small head}}; 
\begin{scope}[text=red]
\node at (7,-0.5) {\text{\small current}}; 	
\end{scope}
\foreach \i in {0,...,4} {
	\def\data{{30, 90, 10, 70, 20}}
	\ifthenelse{\i=2} {
		\filldraw [fill=white!80!green, rounded corners = 8pt, line width = 0.25mm] (3*\i, 0) rectangle (2+3*\i, 1);
	}
	{
		\filldraw [fill=white!80!red, rounded corners = 8pt, line width = 0.25mm] (3*\i, 0) rectangle (2+3*\i, 1);
	}	\draw (1+3*\i,0) -- (1+3*\i,1);
	\pgfmathparse{1+\i}
	\edef\number{\pgfmathresult}
	
	\pgfmathparse{\data[\i]}
	\edef\item{\pgfmathresult}
	\node at (0.5+3*\i,0.5) {\text{\small \item}}; 
	
	\draw [line width = 0.5mm, -{Latex[length=2mm]}] (1.5+3*\i, 0.5) -- (3+3*\i, 0.5);	
}
\node at (15.5, 0.5) {null};
\filldraw [fill=white!80!blue, rounded corners = 8pt, line width = 0.25mm] (6, -2.5) rectangle (8, -1.5);
\draw (7,-2.5) -- (7,-1.5);
\node at (6.5,-2) {\text{\small 50}};
 \node at (7,-3) {\text{\small node}};
	\draw [line width = 0.5mm, -{Latex[length=2mm]}] (7.5, -2) -- (9, 0.5);	
\end{tikzpicture} \egroup

And the node to be inserted should be after \texttt{current}.

\begin{lstlisting}
current.setNext(node);
\end{lstlisting}

\bgroup \tikzset{png export} \begin{tikzpicture}
\node at (1,-0.5) {\text{\small head}}; 
\begin{scope}[text=red]
\node at (7,-0.5) {\text{\small current}}; 	
\end{scope}
\foreach \i in {0,...,4} {
	\def\data{{30, 90, 10, 70, 20}}
	\ifthenelse{\i=2} {
		\filldraw [fill=white!80!green, rounded corners = 8pt, line width = 0.25mm] (3*\i, 0) rectangle (2+3*\i, 1);
		\draw [line width = 0.5mm, -{Latex[length=2mm]}] (1.5+3*\i, 0.5) -- (7.5, -1.5);	
	}
	{
		\filldraw [fill=white!80!red, rounded corners = 8pt, line width = 0.25mm] (3*\i, 0) rectangle (2+3*\i, 1);
		\draw [line width = 0.5mm, -{Latex[length=2mm]}] (1.5+3*\i, 0.5) -- (3+3*\i, 0.5);	
	}	
	\draw (1+3*\i,0) -- (1+3*\i,1);
	\pgfmathparse{1+\i}
	\edef\number{\pgfmathresult}
	
	\pgfmathparse{\data[\i]}
	\edef\item{\pgfmathresult}
	\node at (0.5+3*\i,0.5) {\text{\small \item}}; 
}
\node at (15.5, 0.5) {null};
\filldraw [fill=white!80!blue, rounded corners = 8pt, line width = 0.25mm] (6, -2.5) rectangle (8, -1.5);
\draw (7,-2.5) -- (7,-1.5);
\node at (6.5,-2) {\text{\small 50}};
 \node at (7,-3) {\text{\small node}};
	\draw [line width = 0.5mm, -{Latex[length=2mm]}] (7.5, -2) -- (9, 0.5);	
\end{tikzpicture} \egroup

Thus, the list becomes,
\vskip 0.5cm
\bgroup \tikzset{png export} \begin{tikzpicture}
\node at (1,-0.5) {\text{\small head}}; 
\foreach \i in {0,...,5} {
	\def\data{{30, 90, 10, 50, 70, 20}}
	\filldraw [fill=white!80!red, rounded corners = 8pt, line width = 0.25mm] (2.5*\i, 0) rectangle (2+2.5*\i, 1);
	\draw (1+2.5*\i,0) -- (1+2.5*\i,1);
	\pgfmathparse{1+\i}
	\edef\number{\pgfmathresult}
	
	\pgfmathparse{\data[\i]}
	\edef\item{\pgfmathresult}
	\node at (0.5+2.5*\i,0.5) {\text{\small \item}}; 
	
	\draw [line width = 0.5mm, -{Latex[length=2mm]}] (1.5+2.5*\i, 0.5) -- (2.5+2.5*\i, 0.5);	
}
\node at (15.5, 0.5) {null};
\end{tikzpicture} \egroup

\newpage

The overall code is below.

\begin{lstlisting}
public void add(Node node, int idx) {
	if(idx < 0 || idx > size()) 
		return;
	
	// at the beginning of an empty or non-empty list
	if(index == 0) { 
		node.setNext(head);
		head = node;
	}
	
	Node current = head;
	for(int i=1; i < idx; i++)
		current = current.getNext();
	node.setNext(current.getNext());
	current.setNext(node);
}
\end{lstlisting}

Note that \texttt{idx == size()} refers to adding an item at the end of the list and the above code works for the same.

\newpage

\subsection{Removing an item from an arbitrary index}

If we need to remove an item at a particular index \texttt{idx} (in this example \texttt{idx = 3}), we need to reach the node \textbf{before} the node to be removed. For this, we move forward \texttt{idx-1} times. Then, simply link the current node (at index \texttt{idx-1}) to the node at index \texttt{idx+1}.

Initial state \textbf{(item to be removed in red)}:
\vskip 0.2cm

\bgroup \tikzset{png export} \begin{tikzpicture}
\node at (0,-0.5) {\text{\small head, }}; 	
\begin{scope}[text=red]
\node at (1.2,-0.5) {\text{\small current}}; 	
\end{scope}
\foreach \i in {0,...,4} {
	\def\data{{30, 90, 10, 70, 20}}
	\ifthenelse{\i=0} {
		\filldraw [fill=white!80!green, rounded corners = 8pt, line width = 0.25mm] (3*\i, 0) rectangle (2+3*\i, 1);
	}
	{
		\ifthenelse{\i=3} {
			\filldraw [fill=white!80!red, rounded corners = 8pt, line width = 0.25mm] (3*\i, 0) rectangle (2+3*\i, 1);
			\node at (1+3*\i, -0.5) {\small (to be removed)};
		}
		{
		\filldraw [fill=white!80!blue, rounded corners = 8pt, line width = 0.25mm] (3*\i, 0) rectangle (2+3*\i, 1);
		}
	}
	\draw (1+3*\i,0) -- (1+3*\i,1);
	\pgfmathparse{1+\i}
	\edef\number{\pgfmathresult}
	
	\pgfmathparse{\data[\i]}
	\edef\item{\pgfmathresult}
	\node at (0.5+3*\i,0.5) {\text{\small \item}}; 
	
	\draw [line width = 0.5mm, -{Latex[length=2mm]}] (1.5+3*\i, 0.5) -- (3+3*\i, 0.5);	
}
\node at (15.5, 0.5) {null};
\end{tikzpicture} \egroup
\vskip 0.2cm

After iteration 1:
\vskip 0.2cm

\bgroup \tikzset{png export} \begin{tikzpicture}
\node at (1,-0.5) {\text{\small head}}; 
\begin{scope}[text=red]
\node at (4,-0.5) {\text{\small current}}; 	
\end{scope}
\foreach \i in {0,...,4} {
	\def\data{{30, 90, 10, 70, 20}}
	\ifthenelse{\i=1} {
		\filldraw [fill=white!80!green, rounded corners = 8pt, line width = 0.25mm] (3*\i, 0) rectangle (2+3*\i, 1);
	}
	{
		\ifthenelse{\i=3} {
			\filldraw [fill=white!80!red, rounded corners = 8pt, line width = 0.25mm] (3*\i, 0) rectangle (2+3*\i, 1);
			\node at (1+3*\i, -0.5) {\small (to be removed)};
		}
		{
		\filldraw [fill=white!80!blue, rounded corners = 8pt, line width = 0.25mm] (3*\i, 0) rectangle (2+3*\i, 1);
		}
	}
	\draw (1+3*\i,0) -- (1+3*\i,1);
	\pgfmathparse{1+\i}
	\edef\number{\pgfmathresult}
	
	\pgfmathparse{\data[\i]}
	\edef\item{\pgfmathresult}
	\node at (0.5+3*\i,0.5) {\text{\small \item}}; 
	
	\draw [line width = 0.5mm, -{Latex[length=2mm]}] (1.5+3*\i, 0.5) -- (3+3*\i, 0.5);	
}
\node at (15.5, 0.5) {null};
\end{tikzpicture} \egroup
\vskip 0.2cm

After iteration 2:
\vskip 0.2cm

\bgroup \tikzset{png export} \begin{tikzpicture}
\node at (1,-0.5) {\text{\small head}}; 
\begin{scope}[text=red]
\node at (7,-0.5) {\text{\small current}}; 	
\end{scope}
\foreach \i in {0,...,4} {
	\def\data{{30, 90, 10, 70, 20}}
	\ifthenelse{\i=2} {
		\filldraw [fill=white!80!green, rounded corners = 8pt, line width = 0.25mm] (3*\i, 0) rectangle (2+3*\i, 1);
	}
	{
		\ifthenelse{\i=3} {
			\filldraw [fill=white!80!red, rounded corners = 8pt, line width = 0.25mm] (3*\i, 0) rectangle (2+3*\i, 1);
			\node at (1+3*\i, -0.5) {\small (to be removed)};
		}
		{
		\filldraw [fill=white!80!blue, rounded corners = 8pt, line width = 0.25mm] (3*\i, 0) rectangle (2+3*\i, 1);
		}
	}	\draw (1+3*\i,0) -- (1+3*\i,1);
	\pgfmathparse{1+\i}
	\edef\number{\pgfmathresult}
	
	\pgfmathparse{\data[\i]}
	\edef\item{\pgfmathresult}
	\node at (0.5+3*\i,0.5) {\text{\small \item}}; 
	
	\draw [line width = 0.5mm, -{Latex[length=2mm]}] (1.5+3*\i, 0.5) -- (3+3*\i, 0.5);	
}
\node at (15.5, 0.5) {null};
\end{tikzpicture} \egroup
\vskip 0.2cm

Updating the link:

\bgroup \tikzset{png export} \begin{tikzpicture}
\node at (1,-0.5) {\text{\small head}}; 
\begin{scope}[text=red]
\node at (7,-0.5) {\text{\small current}}; 	
\end{scope}
\foreach \i in {0,...,4} {
	\def\data{{30, 90, 10, 70, 20}}
	\ifthenelse{\i=2} {
		\filldraw [fill=white!80!green, rounded corners = 8pt, line width = 0.25mm] (3*\i, 0) rectangle (2+3*\i, 1);
	}
	{
		\ifthenelse{\i=3} {
			\filldraw [fill=white!80!red, rounded corners = 8pt, line width = 0.25mm] (3*\i, 0) rectangle (2+3*\i, 1);
	}
		{
		\filldraw [fill=white!80!blue, rounded corners = 8pt, line width = 0.25mm] (3*\i, 0) rectangle (2+3*\i, 1);
		}
	}	\draw (1+3*\i,0) -- (1+3*\i,1);
	\pgfmathparse{1+\i}
	\edef\number{\pgfmathresult}
	
	\pgfmathparse{\data[\i]}
	\edef\item{\pgfmathresult}
	\node at (0.5+3*\i,0.5) {\text{\small \item}}; 
	
	\ifthenelse{\i=2} 
	{
		\draw [red, line width = .5mm] (1.5+3*\i, 0.6) -- (2.5+3*\i, 1.5);
		\draw [red, line width = .5mm] (2.5+3*\i, 1.5) -- (5+3*\i, 1.5);
		\draw [red, line width = 0.5mm, -{Latex[length=3mm]}] (5+3*\i, 1.5) -- (6+3*\i, 0.6);	
		\node [text width=2cm] at (4+3*\i, -0.5) {\small unreferenced (deleted)};
	}
	{
		\draw [line width = 0.5mm, -{Latex[length=2mm]}] (1.5+3*\i, 0.5) -- (3+3*\i, 0.5);	
	}
}
\node at (15.5, 0.5) {null};
\end{tikzpicture} \egroup
\vskip 0.2cm

End product:
\vskip 0.2cm

\bgroup \tikzset{png export} \begin{tikzpicture}
\node at (1,-0.5) {\text{\small head}}; 
\foreach \i in {0,...,3} {
	\def\data{{30, 90, 10, 20}}
	\filldraw [fill=white!80!blue, rounded corners = 8pt, line width = 0.25mm] (3*\i, 0) rectangle (2+3*\i, 1);	
	\draw (1+3*\i,0) -- (1+3*\i,1);
	\pgfmathparse{1+\i}
	\edef\number{\pgfmathresult}
	
	\pgfmathparse{\data[\i]}
	\edef\item{\pgfmathresult}
	\node at (0.5+3*\i,0.5) {\text{\small \item}}; 
	\draw [line width = 0.5mm, -{Latex[length=2mm]}] (1.5+3*\i, 0.5) -- (3+3*\i, 0.5);	
}
\node at (12.5, 0.5) {null};
\end{tikzpicture} \egroup


\newpage

\begin{lstlisting}
public Integer remove(int idx) {
	if(idx < 0 || idx >= size()) 
		return null;
	if(idx == 0) { //removing head
		double result = head.getData();
		head = head.getNext();
		return result;
	}
	Node current = head;
	for(int i=1; i < idx; i++)
		current = current.getNext();
	int result = current.getNext().getData();
	current.setNext(current.getNext().getNext());
	return result;
}
\end{lstlisting}

\input{comp125lectureFooter}
